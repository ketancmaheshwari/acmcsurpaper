%
\paragraph{Pegasus/DAX} Pegasus \cite{deelman-peg} is an SWE
well-suited to generate and execute large workflows from abstract DAGs
describing dataflow among interdependent tasks. Abstract description of a DAG
is done via an XML-based language called DAX. The concrete expression takes
into account the properties of application task requirements and underlying
distributed compute resources.  The main advantage of such approach is that a
single abstract workflow description can be mapped to multiple execution
environments based on requirements. DAX describes the logical workflow
components, their inputs and outputs and their links and dependencies in a
Pegasus workflow. These kind of workflows have been shown to enact optimally
using Pegasus' dynamic task, memory and disk management techniques. For
instance, for massively parallel task scheduling, Pegasus provides a task
clustering technique that groups parallel jobs based on fixed parameters such
as number of groups. Additionally, a manager buffers and maintains a fix-sized
window of executable tasks before accepting new tasks.

Large scale workflows from astronomy, biology, and high-energy physics have
been shown to run via Pegasus on distributed computing resources such as
Clusters, Grids and Supercomputers.  Recent developments in Pegasus enables it
to run workflows on cloud environments~\cite{peg-cloud}. Pegasus system is
known to be extendible as illustrated by a recent integration
effort~\cite{peg-hubzero} between Pegasus and the Hubzero
portal~\cite{hubzero}.
