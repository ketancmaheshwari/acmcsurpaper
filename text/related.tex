\section{Related Work}
Surveys and studies on workflows similar to the present one have been done in
the past. Yu and Buyya \cite{yu-buyya:2005} proposed a detailed taxonomy of
SWEs in the previous decade, studying many existing systems for distributed
computing. This taxonomy presented is based, to a large extent on the
capabilities of the workflow engines. It does provide insights to many of the
SWE aspects considered in the present paper, especially the workflow languages
expressiveness and SWEs interactions with the distributed computing
infrastructures in the early days of the compute Grids. A more recent work by
Goderis \cite{goderis-brooks-etal:2009} studies a topic more closely related to
the workflow language expressiveness: it compares the models of computations of
various workflow directors implemented in Kepler
\cite{ludascher-altintas-etal:2005a} with the perspective of enabling
composition of different models in sub-workflows of a global orchestration.
This work illustrates how different enactors designed for different application
needs have different expressiveness power. Deelman et. al.
\cite{deelman-overview} present a high-level overview of the scientific
workflow features and capabilities including SWE composition, representation
and resource mapping capabilities. A recent survey from
Stratan~\cite{stratan-iosup-etal:2008} about the workflow engine performance
takes into account the latencies involved during the enactment mechanisms. A
small survey and analysis of file-access characteristics for data-intensive
workflows has been done by Shibata~\cite{Shibata}. Early surveys on workflow
languages and suitability of general purpose languages to workflows is
conducted in~\cite{generalpurpose}. These surveys were conducted before a few
years and in the meantime SWEs have come a long way in terms of capabilities
and enactment approaches. New techniques such as the map/reduce framework
\cite{mapreduce} have been introduced as a promising new approach to workflow
enactment over distributed infrastructures. Several issues put forth by such
surveys such as customized service wrapping and enactment have been adequately
addressed over the
years~\cite{glatard-montagnat-etal:2008a,rojasbalderrama-montagnat-etal:2010}.

Another well-studied feature is the workflow languages. Many different
languages have been considered within the SWE community, from raw Directed
Acyclic Graphs (DAGs) of computational processes such as CONDOR
DAGMan~\cite{dagman} to abstractions for parallel computations such as Petri
nets~\cite{alt-hoheisel:2005}, data-driven languages such as
Scufl~\cite{turi-missier-etal:2007} and scripting languages such as Swift
script~ \cite{swift}. Glatard's PhD thesis~\cite{glatard:2007} presents a
taxonomy of SWE languages classifying them into functional, graph,
service-oriented and executable representations. A detailed classification and
taxonomy of parallelism patterns in Grid Workflows has been presented by
Pautasso and Alonso in~\cite{pautasso1}. This classification is more generic
while the one presented in Glatard's thesis has emphasis on service based
workflows.  Each of these approaches can be defended through some aspects well
covered in their design: DAGs are convenient for scheduling
\cite{Malewicz:2007,hall-rosenberg-etal:2007}, Petri nets can be used to detect
properties such as potential deadlocks, data-driven languages ease the
description of application logic for non-expert users and scripting is
extensively used by programmers for prototyping, etc.
