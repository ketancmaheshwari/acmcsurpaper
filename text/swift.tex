\paragraph{Swift/Script, Tcl} Swift~\cite{swift} is a script-based scientific
workflow environment developed at the University of Chicago and Argonne
National Laboratory. A Swift workflow specified in the Swift scripting language
and is translated to an intermediate XML-based representation and enacted by
the Karajan workflow enactment engine. The SwiftScript language is proposed as
a workflow language founded on a scripting approach that programmers are
familiar with. To make the language data parallel, Swift script generalizes the
use of futures~\cite{futures} for every variables defined in the
script.  Futures are non-blocking assignment variables, using a proxy to ensure
immediate execution of assignment and performing lazy blocking on variable
value accesses only. The systematic use of futures for every variables in the
language makes it completely data-driven: the execution progresses
asynchronously as long as a data access is not blocking. The availability of
data enables blocked thread to restart execution. Swift facilitates a highly
asynchronous workflow execution. Swift supports advanced workflow constructs
such as parallel loops and conditionals. Furthermore Swift supports processing
of explicitly defined arrays of data. Swift workflows take advantage of
POSIX-style shared file system to reuse the data produced by other processors.
Recently, Swift has been shown to run on multiple clouds~\cite{ccgrid2013} via
its coasters mechanism~\cite{ucc2011}. Swift/T~\cite{turbine2013}, a new HPC
implementation of Swift system, runs scientific workflows over large-scale
supercomputing resources. It has been shown to work on petascale supercomputers
such as the IBM BlueGene and Cray series of machines. GeMTC project
investigating running many-task applications over GPUs via Swift/T. Efforts to develop metadata and provenance modules have been undertaken recently~\cite{swiftprov}.

