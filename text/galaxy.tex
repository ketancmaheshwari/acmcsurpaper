\paragraph{Galaxy Portal/JSON} Galaxy~\cite{blankenberg-vonkuster-etal:2010} is a
SWE developed at the Penn State University, Pennsylvania. It provides a
web-based portal environment to build complex pipelines from an available
repository of processors called tools within Galaxy project
(\textit{http://g2.bx.psu.edu}).  Galaxy is currently used mainly by
bioinformatics community. Users can contribute and extend the tool repository
provided by Galaxy using a hosting server. One of the major use of Galaxy is to
achieve a mix and match of data analysis tools in order to analyze a vast
repository of genome based database. Galaxy offers a web-based, interactive
platform for data analysis via reusable and reproducible workflows.  While
excelling in the usability aspects, the Galaxy environment has a limited
parallelism expression capabilities and a small number of existing
implementations for interfacing with diverse, scalable computing
infrastructures. For instance, it is hard to express the commonly found
\texttt{foreach} parallel expression using Galaxy's visual interface. The user
can draw workflows in which multiple copies of a task are created to operate on
multiple inputs at once. However, workflows tend to get cluttered quickly and
is not scalable to large-scale computations.
