\paragraph{Kepler/MoML} Kepler~\cite{kepler2013} is a SWE
based upon the Ptollemy II framework~\cite{eker-janneck-etal:2003} developed at
the University of California. Kepler workflows follow an actor-director
metaphor where an actor represents a processor and the director
represents the paradigm for executing the workflows consisting of the actors.
Depending upon the types of directors, a Kepler workflow may exhibit a
different type of execution behavior. A workflow designer can chose from among
directors based upon the purpose of a workflow. Few of these include Process
Network (PN) for a parallel execution at the workflow level, Synchronous Data
Flow (SDF) for simple sequential data flow and Collection Oriented Modeling and
Design (COMAD) \cite{keplercomad} for a collection oriented workflow. Kepler
uses the Modeling Markup Language (MoML) for its workflow representation. MoML
is an extended XML syntax providing a rich vocabulary to compose Kepler
workflows. A workflow processor is called an `actor' in Kepler. Recent advances
in Kepler include MapReduce style computation, bioinformatics computation,
cloud computing~\cite{kepler2013}.
