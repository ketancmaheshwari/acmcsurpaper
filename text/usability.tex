\section{Usability} \label{sec:usable}
A workflow environment influences the end-user comfort and convenience towards
the goal of efficiently organizing and deploying a scientific experiment
involving heavy usage of computational resources. However, there is no single
factor that determines the usability of a SWE. A combination of several
qualitative and quantitative factors play a role in determining how usable a
given SWE is.

\begin{table}[ht!]
\begin{center}
% use packages: array
\begin{tabular}{|l|l|l|l|}
\hline
\textbf{SWE/Language} & \textbf{Workflow Composition} & \textbf{Reusable} & \textbf{Extendable}\\
\hline
Taverna/Scufl2 & GUI & \checkmark & \checkmark\\\hline
+/BPEL & GUI & \checkmark & \checkmark \\\hline
Triana/- & GUI & \checkmark & \checkmark \\\hline
MOTEUR/Scufl & GUI & \checkmark & -- \\\hline
Kepler/MoML & GUI & \checkmark & \checkmark \\\hline
GridNexus/GXPL & GUI/Text & -- & -- \\\hline
Swift/SwiftScript & Text & -- & -- \\\hline
Askalon/AGWL & GUI & \checkmark & \checkmark \\\hline
P-GRADE/- & Web Portal & \checkmark & -- \\\hline
Pegasus/DAX & GUI & \checkmark & \checkmark \\\hline
Galaxy/- & Web Portal & -- & -- \\\hline
\multirow{2}{*}{MOTEUR2/} & GUI/Text & \checkmark & --\\
Gwendia~/gscript & & & \\\hline
\end{tabular}
%\caption{A summary of SWEs Usability Analysis}
\label{tbl:usability}
\end{center}
\end{table}

In this section, the following aspects of grid workflow usability are mainly
considered: a) Ease and convenience for workflow composition. This largely
rests on the available composition interfaces such as a GUI. b) Workflow
reusability. c) Extensibility by being able to be linked to external
applications or the ease of customizing the existing codebase of the SWE. Since
legacy applications are not developed keeping workflow enactment in mind, it
largely depends upon the extendability of the workflow environment to adapt to
the application's requirements. We discuss these features in more details in
the rest of this section and summarize them in table \ref{tbl:usability} for
the SWEs studied.

A visual workflow development environment is most prominent among the workflow
community. The visual paradigm of workflow development is convenient as it
provides an intuitive ability to the user to design workflows by visualizing
the flow of data and the processors involved in the data processing at each
stage. It also facilitates users to visualize the opportunities to exploit
parallelism in the graph.

Another dimension of usability is from the point of view of the computational
infrastructure and its middleware. A SWE must optimally expose and make
available all the features of the underlying middleware components.

On the other side of the workflow development paradigm are the script-based
workflow development capability. A script-based workflow development has its
own advantages and can lead to the development of highly expressive, compact
and portable workflows rapidly in the hands of expert or well-experienced
end-users. An ideal situation is a workflow manager allowing both visual as
well as scripting development of scientific workflows.

There are two key aspects to highly usable SWE, expressibility and simplicity.
From the end-user point of view, these are often contradictory and a trade-off
is required to attain a balance between the degree to which each aspect is
contained and come in the way of user comfort. User comfort, time to design and
construct a workflow are of particular importance in the case of SWEs as the
users are not necessarily technical experts.

This section explores the above mentioned aspects and features seen in the
SWEs.

The above mentioned aspects also play a role in workflow reusability.
Scientific procedures are recurring in nature and offer opportunities to reuse
fully or partially , the existing workflows~\cite{xiaorong-madey:2007}. In
order to be able to reuse an existing workflow, it must be easily available and
clearly described as to what are its functions.

Furthermore, other operational factors contribute significantly to a SWE's
overall usability. An example is as reported by Ramakrishnan et. al. in
\cite{ramakrishnan-nurmi-etal:2009} wherein factors such as virtual
reservations, coordinated use of resources and fault tolerance combined
together considerably enhance workflow usability. Among these, the fault
tolerance mechanisms are the most challenging to implement and realize at the
workflow level. This is owing to the fact that workflows interacting with many
diverse entities have a large number of, and unforeseen fault conditions. A
small number of SWEs provide some kind of fault tolerance mechanisms. For
instance MOTEUR provides a `retry' mechanism on job fault while SwiftScript and
BPEL provide an exception handling mechanism in order to notify users of
occurrence of known fault conditions.

Some interesting efforts have been made in the past to increase usability of
grid environments for a particular domain users such as medical imaging
\cite{olabarriaga-deboer-etal:2006,olabarriaga-glatard-etal:2008}. In the case
of SWEs, specific usability issues have been addressed relatively late.
Distributed workflow repositories \cite{stoyanovich-taskar-etal:2010} have been
especially popular recently. These repositories play an important role in
making a large number of scientific workflows available and hence reusable. The
\textit{myexperiment.org} \cite{deroure-goble-etal:2008} is one such popular
repository of Taverna workflows available publicly. Other examples of such
repositories are the workflow libraries maintained as a part of the Kepler
project (library.kepler-project.org/kepler/) and another one with the Vistrails
project \cite{callahan-freire-etal:2006}. Kepler also integrates with DataStaR
\footnote{https://wiki.duraspace.org/display/FEDORACREATE/DataStaR+Use+Case} to
support data sharing among researchers during the research process, and to
promote publishing or archiving data and metadata to discipline-specific data
centers. The workflows hosted in these repositories are tagged and categorized
based upon their features and functionalities. The ``myexperiment" repository
is especially notable for its adoption of social web paradigm where users not
only find and reuse the workflows but can actively collaborate on scientific
experiments over the web.
