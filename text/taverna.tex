\paragraph{Taverna/scufl} The Taverna Workflow Management System
\cite{taverna2013} is an open source scientific dataflow manager
developed by the mygrid~consortium in the UK.
%
Taverna includes a GUI-based rich-client workbench for workflow design and a
workflow enactment runtime. Taverna can enact workflows expressed in the scufl
language. Scufl is a near Turing complete dataflow language with well-defined
semantics \cite{turi-missier-etal:2007,glatard-montagnat:2008}. 

A Taverna workflow consists of a collection of processors connected by data and
coordination links, which establish a dependency between the output(s) of a
processor and the input(s) of another. The processors are of different kinds
depending on the application code to be invoked: typically being services, with
input and output \textit{ports} that correspond to the operation's invocations
and response sequences defined in the service`s interface expressed as REST or WSDL.
Java classes and local scripts can be used as workflow processors, as long as
they expose their signature as input and output ports as required by the model.
The engine orchestrates the execution of the processors in a way that is
consistent with the dependencies, and manages the flow of data through the
processors' ports. The overall workflow execution is data-driven, with a
\emph{push} model: a processor's execution is started as soon as all of its
inputs are available. Taverna is a flexible and extendible workflow environment
\cite{missier-soiland-reyes-etal:2010,sroka2009a} in that extending the
engine's functionality is facilitated through a Service Provider Interface
(SPI). The Taverna engine implements a sophisticated mechanism for enactment of
parallel workflows. These mechanisms manifest in the form of complex pipelines
and an advanced superpipelined enactment of workflows. Such a mechanism induces
multiple downstream workflow processors running in parallel for a dataset. In
the case of superpipelines, multiple pipeline instances run for multi-datasets
providing a high degree of parallelism limited only by the availability of
underlying resources. Taverna is not natively interfaced with external
computational infrastructure but the same can be achieved with the help of
extending the engine through one of its
SPIs~\cite{maheshwari-missier-etal:2009}. In a crucial development, recent
implementations of Taverna offers a client-server based environment in which a
Taverna \emph{server} can remotely serve the workflow composition and execution
services. This mode is useful in the cloud computing environments where the
server decides where to run the Taverna services in the cloud.
