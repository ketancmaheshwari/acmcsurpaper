\section{Objectives and Organization}
% What are the goals of this work?
% How is this paper supposed to help:
% -- Existing scientific users who have not adopted workflows
% -- Existing users who have adopted workflows
% -- New users
% -- Experts in the field

% Say that different workflow systems have different design goals. It is hard
% to chose one from the existing or build your own. It is almost equally
% complex to build own just as it is complex to create own programming
% platform. We hope to provide by the way of this paper a guide for new adopters on how to chose a workflow language appropriate for their needs. We ask and answer the following questions of the workflow systems that we study in this paper:
%-- What were the original design goals?
%-- What are the main application areas that the system is currently being used?
%-- What are the characteristics and capabilities of the language? expressivity features?
%-- What are the characteristics and capabilities of the run-time? parallelism? concurrency? task and data grouping?
%-- What kinds of computing and computational infrastructure does the workflow system cater to? Grids, Clouds, Clusters, Supercomputers, fast IB networks? Accelerators?
%-- How have they evolved? What are the possible future directions?

In the present work we survey and study the various contemporary SWEs in order
to find the capabilities and key features in terms of the characteristics
described in section~\ref{sec:intro}. We believe this study will serve to
address the following objectives:

\begin{enumerate}
\item \textbf{Educate.} With more science being done with the help of
computational methods, SWEs can add significant value to the process.  They can
accelerate the process, enable community driven science, drive the development
at a manageable level of abstraction for users.  However, scientific users are
often not experienced programmers or computation savvy. This work will serve as
a useful starting guide to educate science users and students of the available
tools and techniques with features suitable for their methods.

\item \textbf{Catalog.} We study a number of important features as offered by
the contemporary SWEs. This will help catalog and collect the SWEs by their
offerings and identify mechanisms and needs for new and existing features
in one place. This will also help identify current trends and approaches
to computation in scientific communities.

\item \textbf{Application.} We identify the applications and application
classes well adapted to the workflow systems and are benefitting by them.
This will help readers relate to the application they are associated to.
We hope that readers can make a choice of which SWE to use if there is an
adaptation similar in scope to the ones describe in this paper.

\item \textbf{Evolution.} We comment on the evolution of the existing workflow
systems and the possible current and future development directions which
will help readers be aware of the general development direction of the
SWEs.
\end{enumerate}

The rest of this paper is organized as follows: Section
\ref{sec:overview} presents an overview of the key SWEs studied in this work.
Section \ref{sec:scal} discusses the optimization features and resulting
scalable performance as observed in the SWEs studied. Section \ref{sec:data}
studies the data management and description aspects of the SWEs. Section
\ref{sec:inter} presents a study of SWEs approaches to interface the
large-scale computing infrastructures. Section \ref{sec:lang} details the
language features and expressiveness of the SWEs studied. Section
\ref{sec:usable} discusses the usability aspects of the SWEs.  Section
\ref{sec:sum} presents a summary and a classification of all the aspects of
SWEs studied in this work. 

Throughout the paper, we also present tables
summarizing the aspects of SWEs considered in respective sections. It was not
possible to obtain an exhaustive information about all the SWEs considered in
this work. We rely on the following sources for our information: 
\begin{enumerate}
    \item \textbf{Publications.} Peer reviewed publications in journals, conferences and workshops.
    \item \textbf{Technical Reports.} Technical reports and theses.
    \item \textbf{Web Presence.} Documentation and other information from the web pages hosting the SWE.
    \item \textbf{Experience.} Where possible, we tried features of the SWEs by installing them on the available systems. 
\end{enumerate}

Some features, such as extendability of an SWE strongly depend on the extension
points designed in the code; consequently, we consider it present in only those
SWEs that have an explicit documented description or demonstration of the given
feature. 
