% == DATA MANAGEMENT ==
\section{Data Management and Description} \label{sec:data}
Many aspects of data handling present significant diversity among the workflow
managers. Data management approaches for scientific workflows can be classified
as resource and language level. Resource level strategies include data
management at the application level of abstraction where the data is required
to be distributed, available and securely remote-accessible. It also determines
how the data will be delivered to a resource that handles the processing and
returning the processed data or results to the next processors in the workflow.
Provenance, staging, and managing heterogeneous data structures (files, arrays
etc.) are functional requirements at this level.

Whereas the language level data management involves how the actual data items
are described and handled across a workflow execution. Decisions such as how
complex data structures are supported, transferred and delivered from workflow
input source or from one processor to another are determined at language level.
In most workflow systems, the transfer of data is achieved through `data-ports'
attached to the processors. These ports act as channels to collect and transfer
data across a workflow. The association of data description with the workflow
determines whether the data is loosely or strongly coupled with a workflow. A
loosely coupled workflow with data has advantage over the strongly coupled ones
in the sense that the workflow is independent of its data and can be invoked
with a different set of data.

A source of complexity for the workflow manager is the existence of different
formats of data for a single scientific experiment. These include the different
legacy formats of storage as well as the different ways the data is organized
including database, filesystems and remote links. Further, most remote data is
stored at locations secured behind different types of secure communication
protocols involving special access mechanisms.

In the rest of this section, we study the data handling mechanisms of SWEs.
Taverna's reference scheme makes sure the data produced and consumed is in the
form of references. This approach of data management is efficient because: 1)
It takes less time to transport references, and 2) Locality of references
ensures that the data is processed at a location that is local to multiple
processors operating on the same dataset.

In MOTEUR and Kepler workflow environments, data is described as an XML grammar
allowing for nested collections or multi-dimensional arrays of data. Kepler's
COMAD \cite{keplercomad} workflow enactment model is well suited to collection
oriented data like arrays and streams. Similar efforts exist elsewhere to model
the streams for workflow programming
\cite{blower-haines-etal:2005,herath-plale:2010}.

MOTEUR only uses references to data files while Pegasus groups small files into
a single compressed file for transfer thus saving transfer time. It optimizes
the workflow by annotating and reusing the data already available, if
appropriate. Pegasus is able to generate a workflow from a metadata description
of the desired data product with the aid of Artificial Intelligence planning
techniques. Grouping of tasks by Pegasus increases the data usage efficiency
since a group of tasks that require the same set of files for execution are
localized. Pegasus also manages the temporary storage space needed for data
processing at a particular processor location and commands the local Operating
System to allocate the space. A data provider abstraction allows data to be
referenced in the Trident SWE from any external entities in the form of typed
referenced data objects.

A majority of SWE support primitive, MIME based or XSD data types similar to
traditional programming datatypes. Specific support for complex and structured
data types are provided by the SWEs based upon their domain and application
requirements. For instance, Kepler supports a specialized byte data streams and
Bioflow supports a natively developed life-science database engine. Others such
as the SwiftScript support extensible structures such as filesystem mappers and
customized C language style structures.

Remote and secure data access is supported by SWEs by providing secured
protocols based accession to remote data repositories. Third party protocols
such as GridFTP, HTTP over TLS (HTTPS) and gLite LFC (LCG File Catalog) access
are commonly supported.

\begin{table}[ht!]
\begin{center}
\hspace*{-2.2cm}
% use packages: array
\begin{tabular}{|c|c|c|c|c|c|}
\hline
\textbf{SWE /} & \multicolumn{3}{c|}{\textbf{Language Level}} & \multicolumn{2}{c|}{\textbf{Resource Level}}\\
\cline{2-6}
\multirow{2}{*}{\textbf{Language}} & \textbf{Data Coupling} & \textbf{Data Types} & \textbf{Data Structures} & \textbf{Type} & \textbf{Access}\\
& \textbf{with Workflow} & \textbf{Supported} & \textbf{Supported} & & \\
\hline
Taverna/Scufl2&Loose&MIME-types&multi-dim Arrays&Local+Remote&By Reference\\\hline
+/BPEL&Strong&XSD-Based&--&Grids, Clouds& Adhoc\\\hline
%
\multirow{2}{*}{Triana/-}&Strong&XML Schema-&Streams,&Local+&Through \\
&&Types&Java Objects&Remote&Adaptors\\\hline
%
\multirow{2}{*}{MOTEUR/Scufl}&Loose&MIME-&Lists&Local+&Services+\\
&&Types&&Remote&Middleware\\\hline
% 
\multirow{2}{*}{Kepler/}&Strong&Primitive&Collections,&Filesystem&Customized\\
MoML&&Types&User defined&&Actors\\\hline
% 
GridNexus/GXPL&Loose&XSD-Based&--&Grid Data&OGSA-DAI \\\hline
% 
\multirow{2}{*}{Swift/SwiftScript} &Strong&--&1-Dim Arrays,&Shared& POSIX-style\\
&&&Customised&FileSystem&\\\hline
% 
Askalon/AGWL&Loose&MIME-types&Arrays&Data Repository&SQL-based\\\hline
%
\multirow{2}{*}{P-GRADE/-}&Loose&--&--&Grid&Secure\\
&&&&File Catalogs&Protocols\\\hline
%
\multirow{2}{*}{}Pegasus/&Loose&--&--&Distributed&Secure\\
DAX&&&&Filesystems&Protocols\\\hline
% 
Galaxy/-&Loose&--&File Streams&File Systems&--\\\hline
% 
\multirow{2}{*}{MOTEUR2/} &Loose&Integer,String,&multi-dim Arrays&Grid&By\\
Gwendia~ &&Double,URI&&Data&Reference\\\hline
\end{tabular}
%\caption{A summary of SWEs Data Management Mechanisms}
\label{tbl:datadesc}
\end{center}
\end{table}

Table \ref{tbl:datadesc} presents a summary of the data management mechanisms
of the studied SWEs. Language and resource level data management mechanisms are
highlighted. At the language level, data description as coupled with that of
the workflow, data types and structures are considered for the SWE studied. At
the resource level, the major resource types supported and their access
mechanisms are highlighted. We distinguish the remote form of Grid resource
types by the specific Grid protocols (\textit{eg.} GridFTP) supported for those
resource access. Other resource types such as data repositories or file
catalogues are the resource types specific to a given SWE. For instance,
Askalon uses a specifically configured data repository to access data and
application binaries.

%Replication is one of the solution.
