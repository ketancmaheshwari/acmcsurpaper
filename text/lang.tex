
% == WORKFLOW LANGUAGE EXPRESSIVENESS ==
\section{On Workflow Languages} \label{sec:lang}
A workflow language is the medium of expression of the task a workflow is
supposed to accomplish. A workflow language acts as an interface between the
workflow enactor and an end-user. Apart from correctly expressing the flow of
execution and data, a workflow language must be able to link with the
application code in a transparent way.

Historically, low-level, general-purpose scripting languages have been used for
workflow descriptions. For example, batch files, UNIX shell scripts and
script-based languages like perl, python \cite{begeman-belikov-etal:2010} etc.
There also exist some domain specific scripting languages such as bioperl
(\textit{www.bioperl.org}). However, these languages are less suitable for
expression of high-level, sophisticated workflow constructs such as iterations,
while at the same time dealing with parallelism and array processing. They also
fall short of providing implicit parallelism which must be defined explicitly by
the user designing workflows. Description of remote data is another important
aspect that these languages are limited in expressing. Thus, in modern SWEs, it
is required to have a higher-level, specialized and dedicated workflow
language.

Applications and in-silico scientific experiments exhibit diverse patterns of
data and/or control flow. Apart from this, operations and combinations of data
are required to correctly and efficiently express a scientific workflow. This
has led to a requirement of highly expressive workflow languages. Most tasks
could be accomplished by simplified and minimal constructs in a language,
however, they lead to unintuitive and inefficient workflow design. A
sophisticated language not only increases the readability and maintainability
of a workflow, it also increases the chances of its efficient deployment and
execution in the target computational environment.

Description of data and dataflow in a workflow such that the application code
is invoked correctly and in accordance with the experiment has been a
challenging task for large experiments. This is partly because of the large
amount and different type of parallelization opportunities and partly because
of the way the components of an application deal with the data.

A workflow language must be able to express the invocation that is required for
tools to deploy the application components. For example, an experiment may be a
combination of remotely executable services and deployable legacy application
binaries. Along with the application binaries, a workflow language must be
capable of handling the different expressions for different kinds of data
transfer methods and protocols. A common way of invoking legacy applications is
through a traditional Command Line Interface. It is thus desirable to have a
way to wrap such command lines in order to provide a suitable interface to such
a legacy
application~\cite{rojasbalderrama-montagnat-etal:2010,maheshwari:2007}.

%
\begin{table}
\begin{center}
% use packages: array
\begin{tabular}{|l|l|l|l|}
\hline
\textbf{SWE/Language} &\textbf{Control Structures}& \textbf{Array Programming} & \textbf{Iterators}\\
\hline
Taverna/Scufl2 &Conditionals and Loops& \checkmark & \checkmark \\\hline
+/BPEL &Conditionals and Loops& -- & -- \\\hline
Triana/- &Conditionals and Loops& \checkmark & -- \\\hline
MOTEUR/Scufl &Conditionals& -- & \checkmark \\\hline
Kepler/MoML &Conditionals and Loops & \checkmark & limited \\\hline
GridNexus/GXPL &Conditionals and Loops& -- & -- \\\hline
Swift/SwiftScript & Conditionals and Loops & 1-dim & limited \\\hline
Askalon/AGWL & Conditionals and Loops & -- & -- \\\hline
P-GRADE/- & Conditionals & -- & \checkmark \\\hline
Pegasus/DAX & -- & \checkmark & -- \\\hline
Galaxy/- & -- & -- & -- \\\hline
MOTEUR2/Gwendia~ & Conditionals and Loops & \checkmark & \checkmark \\\hline
\end{tabular}
%\caption{A summary of SWEs Expressiveness Capabilities}
\label{tbl:expressive}
\end{center}
\end{table}

\paragraph{Dataflow and Functional Languages} Functional and Dataflow languages
and their derivatives have been a viable choice for workflow languages
paradigm. In the dataflow paradigm, a workflow processor will `fire' when the
data sufficient for its firing is available. Dataflow programming paradigm
particularly suites the workflow approach of programming. There are many
research works advocating a dataflow oriented approach to solve problems
involving a flow of data among computing components, predominantly in a
Directed Acyclic Graph form \cite{dataflow,lee-messerschmitt:1987}. Early
research on dataflow languages as visual language~\cite{sutherland:1966}
reinforce the notion of the suitability of dataflow paradigm for workflows.

Functional programming complement the dataflow paradigm by providing
appropriate expression and semantics to a dataflow. Several features of
functional programming are shared by the dataflow programming approach. One of
the key hinderance of functional programming is the `side-effects' produced
when the values of variables are mutated more than once during a single
execution. Dataflow and functional programming approach converge when these
`side-effects' are eliminated and a functional program can express pure
dataflow under such conditions. 
% Modify and write ==> Functional languages, such as SCALA and Haskell, offer
% an implicit way of exposing parallelism, making it easier to identify data
% dependencies so that the runtime system can implicitly schedule independent
% work (work that respects the dependency graph).
In this section we study the features of contemporary workflow languages that
make them expressive of representing complex data and control flow situations
and enable them to map these expression onto the underlying computing
infrastructure.

Pegasus uses DAX which is an XML-based language to describe DAGs. Applications
using Pegasus use higher level languages such as Python or Java to describe
workflows and then map them onto DAX specification. Such transformations face
two issues: 1. DAGs are simple to transform but can get complicated if
hierarchical DAGs are used. 2. DAGs representations need sophisticated
translators. All features in low-level languages such as loops are not
supported by DAGs. This must be taken care of by the translators.

Kepler, GridNexus and Triana support loops, conditionals and sub-workflows as
advanced constructs of their workflow language. These constructs are manifested
in the GUI as configurable components that get their mapping in the underlying
language by a mapper software component.

ASKALON's AGWL (Abstract Grid Workflow Language) provides an abstract XML-based
representation of dataflows. It does not directly support control flow, however
control flow can be achieved indirectly using `pseudo datatokens'. It is
modularized and supports sub-workflows. AGWL provides conditionals, iterators,
and explicit parallelism. AGWL provides sophisticated workflow constructs
including sub-workflows, parallel and sequential execution constructs, loops,
conditionals and iteration strategies.

The GWorkflowDL employed by GWES is an XML-based representation of Workflows
which are based on the theory of high-level petri nets. This theoreticaly
well-found language provides basis for high expressiveness to GWES workflows.
It leads to the expressions of workflows at three different levels of
abstractions from user-intentions at high-level to the executable services
instances at the low-level.
%Swift is dynamic
%
%A paragraph saying the most generic language approaches of contemporary
%workflow environments
Both Askalon and Pegasus enable users to define abstract workflows to be
concretized at the later stage of workflow development. This feature
facilitates users to have the flexibility to generate concrete workflows based
upon the state and availability of resources at the time of enactment. 

The graphical representation is the only workflow language in Knime. It
provides an interesting feature of metanodes that acts as a self-contained
subworkflow. A metanode is a processor that act as a container for other
processors. Thus a branch of a workflow in Knime can be configured into a
single processor in the form of metanode. A single such processor might be
enacted multiple times giving a form of loop to a Knime workflow.

Table \ref{tbl:expressive} shows a summary of language expressiveness
capabilities of the SWEs studied. We consider the control structures:
conditionals and loops, support for arrays and iterators
\cite{montagnat-glatard-etal:2006} as the expressive features of SWEs.
