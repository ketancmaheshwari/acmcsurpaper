
% -------------------- Fit here as not main content ------------------------
\subsection{Other Systems}
In this section we briefly discuss various other systems which are either too
niche or obsoleted.

Business Process Execution Language (BPEL)~\cite{bpel} is a commercil workflow
platform pioneered by IBM Corporation and used predominantly to define business
workflows. It is now a broad standard for business workflow applications. BPEL
is an open language and commercial enactors are available that provide
enactment functionality for the BPEL workflows. There has been attempts to use
BPEL or its extended version in the scientific applications. However, the
language does not suit well to the data-intensive, grid workflows and has a
limited adoption in the scientific
community~\cite{tan-missier-etal:2010,barga-gannon:2007,scherp-hoing-etal:2009}.
Apart from that, a lack of an open-source workflow engine with graphical
interface also contributes to its limited acceptance in academic research.
However, BPEL has been famously unsuitable for data intensive scientific
applications  BPEL lacks different user-defined strategies to effect multiple
invocations of each processor according to the application semantics. BPEL also
lacks data streams and implicit parallelism forcing a user to explicitly define
parallelism using one of its parallel processing constructs such as
\texttt{foreach}.

%
The Grid Workflow Execution Service (GWES)~\cite{Hoheisel2007} is a SWE
developed by the Fraunhofer-First institute. It implements a dynamic workflow
composition and enactment model based on the Grid Workflow Description Language
(GWorkflowDL)~\cite{Alt2006}.  GWES enables key workflow development and
deployment lifecycle phases including visual composition, data-staging, Grid
middleware interfacing and execution monitoring. Currently, GWES is intensively
being used in the field of Medical Image Processing especially at the
Charit\'{e} Hospital at Berlin in Germany.  It has been successfully interfaced
with the German D-Grid infrastructure.

%
The GridNexus \cite{GridNexus} SWE, developed at the University of North
Carolina, Wilmington, facilitates a graphical composition and execution of
DAG-based workflows. Like Kepler, GridNexus is also based on Ptolemy II
framework. GridNexus workflows are expressed in an XML-based scripting language
called JXPL \cite{hunt-ferner-etal:2005}. JXPL is a customizable and extensible
scripting language allowing users to write their own workflow processor types.
GridNexus workflows are grid oriented and can be deployed on grid resources
accessed through a persistent WS-RF (Web Services Resource Framework)
interface.

%
Bioflow~\cite{bioflow} is a query based workflow language.
It is motivated by the fact that bioinformatics applications are query based
applications that benefit from the complex database queries or saved database
`views'. A typical Bioflow workflow pipelines a sequence of filter queries in
order to get the desired results. Bioflow is mainly utilized in the area of
bioinformatics and querying genome databases. Additionally, a Bioflow workflow
provides mechanisms for data manipulation and updation in the existing
databases. It also includes advanced features such as parallel processing of a
set of mutually independent database queries. The execution model is based on
the popular database query language SQL (Structured Query Language). Processes
can be defined in Bioflow which are processor counterparts in workflow paradigm
and similar to stored-procedures in SQL jargon.

%
Vizbuilder~\cite{vizbuilder} is a graphical toolkit to generate the graphical
representation of bioflow workflows. A Vizbuilder workflow links to an
underlying bioflow engine and provides an interface to submit the bioflow
workflows for execution.

%
Martlet~\cite{martlet} is a text-based, declarative
workflow language that is inspired by the functional programming paradigm. A
Martlet workflow conveniently enables distributed parameter sweep applications
wherein the language intuitively defines the computation and a matching with
the distributed resources can be carried out at workflow execution time. List
processing applications where a sequence of application components work over an
array of data in a Multi-data multi-instruction style setup is well-suited to
the Martlet's model of computation.

%
Knime~\cite{knime} is a distributed, parallel pipelining
system for data intensive applications developed at the University of Konstanz.
It is a java-based platform with a Graphical Interface that is based on eclipse
(\textit{http://www.eclipse.org}) plug-in framework. A Knime workflow consists
of processors belonging to categories such as readers, manipulators, learners,
predictors, etc. Each processor has a configuration dialog associated with it
that allows user to set its parameters. Processors are connected through data
and/or control connections. A highlight of knime is support for shared- and
distributed-memory workflow enactment and support for parallel workflow loops
and subworkflows. The loops in Knime manifest themselves in the form of
subworkflows (called metanodes in Knime) by a repeated enactment of the same
subworkflow.

%
StarFlow~\cite{angelino-yamins-etal:2010} is a Python-based
script oriented workflow language used to encode parallel pipelines. StarFlow
handles data as well as control flow using a combination of static, dynamic and
user annotations analysis. A parallelization engine uses data and control
dependencies in a workflow in order to build a parallel profile of the
workflow. StarFlow has been shown to be successfully interface with the SunGrid
\cite{gentzsch:2001} Schedular and Amazon EC2 cloud infrastructure \cite{ec2}.
The key feature of StarFlow is its capability to execute workflows to maximum
possible parallelism. While StarFlow provides a script-based compact and
semiautomatic workflow composition environment, currently it lacks a
sophisticated GUI for workflow composition. 

% ---------------------------------------------------------------------------
